\documentclass[10pt]{beamer}

\usetheme[progressbar=frametitle]{metropolis}
\usepackage{appendixnumberbeamer}

\usepackage{booktabs}
\usepackage[scale=2]{ccicons}
\usepackage[export]{adjustbox}
\usepackage{algorithm, algorithmic}
\usepackage{epigraph}

\usepackage{pgfplots}
\usepgfplotslibrary{dateplot}

\usepackage{xspace}
\graphicspath{{figures/}}

\title{CS 170}
\subtitle{Hashing and Streaming}
\date{}
\author{}
\institute{}

\begin{document}

\maketitle

% -----------------------------------------------------------------------
\section{Hashing}

\begin{frame}{Refresher}

\begin{itemize}
\item \textbf{Hash function:}
$$h: U \to \{0, ..., m - 1\}$$
\item \textbf{Universal hash family $\mathcal{H}$:}
$$\forall~k_1 \neq k_2 \in U,\:\:\: \Pr_{h \in \mathcal{H}}[h(k) = h(k')] \leq \frac{1}{m}$$
\item \textbf{Perfect hashing:} \\
use universal hashing in two layer scheme to achieve zero collisions
\end{itemize}

\end{frame}

\begin{frame}{Hashing 1}

Suppose a hash function
$h: \{0, 1, \ldots, m - 1\} \to \{0, 1, \ldots, m - 1\}$
is chosen from a universal hash family. Then
$$\Pr[h(2) = 2 \cdot h(1) \mod m] = \text{ ?}$$

\end{frame}

\begin{frame}{Hashing 1 {\color{gray} Solution}}

Suppose a hash function
$h: \{0, 1, \ldots, m - 1\} \to \{0, 1, \ldots, m - 1\}$
is chosen from a universal hash family. Then
$$\Pr[h(2) = 2 \cdot h(1) \mod m] = {\color{blue} \frac{1}{m}}$$

\end{frame}

\begin{frame}{Hashing 2}

Let $\mathcal{H}$ be the set of all functions
$h: \{0, 1, \ldots, m - 1\} \to \{0, 1, \ldots, m - 1\}$.

\begin{enumerate}[(a)]
\item Is $\mathcal{H}$ universal?
\item How many random bits are needed to sample a function from $\mathcal{H}$?
\end{enumerate}

\end{frame}

\begin{frame}{Hashing 2 {\color{gray} Solution}}

Let $\mathcal{H}$ be the set of all functions
$h: \{0, 1, \ldots, m - 1\} \to \{0, 1, \ldots, m - 1\}$.

\begin{enumerate}[(a)]
\item Is $\mathcal{H}$ universal?

{\color{blue}
Since $\Pr[h(x) = h(y)] = \sum_{i = 0}^{m - 1} \frac{1}{m^2}
= \frac{1}{m}$ (for $x \neq y)$, $\mathcal{H}$ is universal.}

\item How many random bits are needed to sample a function from $\mathcal{H}$?

{\color{blue}
The size of $\mathcal{H}$ is $m^m$, so we need $m \log m$ bits to sample a function.}
\end{enumerate}

\end{frame}

\begin{frame}{Hashing 3}

Given a prime $p$ and $a, b \in \{0, \dots, p-1\}$, define the function
$h_{a, b}(x) = ax + b \bmod p$ where $x \in \{0, \dots, p - 1\}$.
\textbf{Show that $\mathcal{H} = \{h_{a, b}\}_{a, b \in \{0, \dots, p - 1\}}$
is a pairwise independent hash function family}, i.e. show that for
every $x \neq y$ and $c, d \in \{0, \dots, p - 1\}$,
\[
\Pr_{h_{a, b} \gets \mathcal{H}}\Big[\,h_{a, b}(x) = c \wedge h_{a, b}(y) = d\,\Big] = \frac{1}{p^2}
\enspace.
\]
The notation $h_{a, b} \gets \mathcal{H}$ means that $h_{a, b}$ is chosen uniformly
at random from $\mathcal{H}$ (in other words, $a$ and $b$ are chosen independently
uniformly at random from $\{0, \dots, p-1\}$).

\end{frame}

\begin{frame}{Hashing 3 {\color{gray} Solution}}

{\color{blue}
\textit{All equations are $\bmod~p$.}

$h(x) = c$ and $h(y) = d$ iff $ax + b = c$ and $ay + b = d$.
This is true iff $a, b$ solve the system of equations
\begin{align*}
ax + b &= c \\
ay + b &= d
\end{align*}
Solving this, we have
\begin{align*}
a &= (c - d)(x - y)^{-1} \\
b &= (cy - dx)(y - x)^{-1}
\end{align*}
We are guaranteed that the multiplicative inverses exist because $p$ is prime,
and we know $x \neq y$. Thus there is only one value of $a$ and one value of $b$
that satisfy these equations. Since $a$ and $b$ are chosen independently at random,
the probability of this occurring is $1 / p^2$.
}

\end{frame}

% -----------------------------------------------------------------------
\section{Streaming}

\begin{frame}{Refresher}

\begin{itemize}
\item Have data stream $S = \{x_1, ..., x_m\}$ of unknown length
\item \textbf{Streaming algorithms} process streams $\rightarrow$ give useful information
  \begin{itemize}
  \item Should be single-pass and use a small amount of space
  \item Three components: \textit{initialization}, \textit{processing}, and \textit{output}
  \end{itemize}
\end{itemize}

\end{frame}

\begin{frame}{Stream Sampling}

Given a stream of integers of unknown length, how do you pick one at random
while using no more than two integers' worth of storage?

Every data point should have a $1 / N$ chance of being selected,
where $N$ is the total length of the stream.

\end{frame}

\begin{frame}{Stream Sampling {\color{gray} Solution}}

Given a stream of integers of unknown length, how do you pick one at random
while using no more than two integers' worth of storage?

Every data point should have a $1 / N$ chance of being selected,
where $N$ is the total length of the stream.

{\color{blue}
Store two integers, $x$ (the result) and $n$ (the length of the stream so far).
For each new data point $x_i$, set $x$ to $x_i$ with probability $1 / n$. At the end of the day, return $x$.

The probability of item $i$ surviving through the $n$th time step (for $i \leq n$) is
$$\left(\frac{1}{i}\right)
\cdot \left(\frac{i}{i + 1}\right)
\cdot \left(\frac{i + 1}{i + 2}\right)
\cdots \left(\frac{n - 2}{n - 1}\right)
\cdot \left(\frac{n - 1}{n}\right) = \frac{1}{n} \enspace.$$
}

\end{frame}

\begin{frame}{Document Comparison}

You are given a document $A$ and then a document $B$, both as streams of words.
Find a streaming algorithm that returns the degree of similarity $\frac{|I|}{|U|}$ between
the words in the documents, where $I$ is the set of words that occur in both $A$ and $B$,
and $U$ is the set of words that occur in either $A$ or $B$.

Clearly explain your algorithm and briefly justify its correctness and memory usage
(at most $O(\log (|A| + |B|))$. How accurate is its output?

\end{frame}

\begin{frame}{Document Comparison {\color{gray} Solution}}

{\color{blue}
Simply use the $F_0$ streaming algorithm on $A$, $B$, and $C := A \cup B$
(for this third stream, process words in $A$ and words in $B$). Let $|A|, |B|,$ and $|C|$
be the output of the algorithm on the corresponding set (our estimate for the number
of distinct words in it). Then our estimate for $|I| / |U|$ is $(|A| + |B| - |C|) / |C|$.

Memory usage is logarithmic in the length of the documents, as we're using a constant
number (three) copies of the streaming algorithm shown in class, which used logarithmic
memory. Correctness flows from the correctness of the streaming algorithm for $F_0$,
combined with the set theory axiom $|A \cap B| = |A| + |B| - |A \cup B|$.

The accuracy of the underlying $F_0$ algorithm (and by extension our algorithm)
can be boosted to any level we desire.
}

\end{frame}

\end{document}
