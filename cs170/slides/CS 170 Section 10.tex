%% Requires compilation with XeLaTeX or LuaLaTeX
\documentclass[10pt, xcolor={table, dvipsnames}, t]{beamer}
\usetheme{UCBerkeley}
\graphicspath{{figures/}}

\title{CS 170 Section 10}
\subtitle{Search Problems and Intractability}
\author{Owen Jow\vspace{-1.5ex}}
\institute{owenjow@berkeley.edu\vspace{1.5ex}}
\date{4/04 Algorithm Not Found}

\begin{document}

\begin{frame}
\titlepage
\end{frame}

% -----------------------------------------------------------------------
\section{Overview}

% -------
\begin{frame}{Search Problem}

\begin{itemize}
\item Find a solution $S$ to the problem instance $I$.
\item A solution can be verified in polynomial time by the algorithm $C(I, S)$.
\end{itemize}

\begin{block}{Examples}
\textbf{SAT:} find a satisfying truth assignment for a Boolean formula. \\
\textbf{TSP:} find a tour \footnote{a cycle that passes through every vertex exactly once \vspace{7mm}} of total distance $b$ or less.
\end{block}

\end{frame}

% -------
\begin{frame}{Optimization Problem}

\begin{itemize}
\item Find the \textbf{best} solution $S$ to the problem instance $I$.
\item "Best" should be quantified by some objective function.
\end{itemize}

\begin{block}{Examples}
\textbf{MAX-SAT:} find the max number of clauses that can be simultaneously true. \\
\textbf{TSP-OPT:} find a tour of minimum distance.
\end{block}

\end{frame}

% -------
\begin{frame}{Search vs Optimization}

\begin{itemize}
\item Search and optimization formulations are of equal difficulty.
\item Why? \textit{Each reduces to the other.}
\end{itemize}

$$\textbf{TSP}\ \longleftrightarrow\ \textbf{TSP-OPT}$$

\end{frame}

% -------
\begin{frame}{P vs NP}

\begin{itemize}
\item \textbf{P:} all search problems that can be solved in polynomial time
\item \textbf{NP:} all search problems (i.e. ``verifiable in polynomial time")
\item \textbf{NP-complete:} the problems to which all search problems reduce
\item \textbf{NP-hard:} ``at least as hard as the NP-complete problems"
\end{itemize}

\epigraph{I know an NP-complete joke, but once you've heard one you've heard them all.}{\textit{jason}, Stack Overflow}

\end{frame}

% -------
\begin{frame}

\vspace{3mm}
\includegraphics[width=\textwidth]{np}

\end{frame}

% -------
\smallframetitle
\begin{frame}{Examples}

\begin{table}
\centering
\begin{tabular}{l r}
\tableheadrow
\tableheadcol{NP-complete} & \tableheadcol{P} \\
\texttt{3SAT} & \texttt{HORN SAT} \\
\texttt{TSP} & \texttt{MST} \\
\texttt{ILP} & \texttt{LP} \\
\texttt{RUDRATA PATH} & \texttt{EULER PATH} \\
\texttt{BALANCED CUT} & \texttt{MINIMUM CUT} \\
\texttt{LONGEST PATH} & \texttt{SHORTEST PATH}
\end{tabular}
\caption{``Hard" versus ``easy" search problems.}
\end{table}

\end{frame}

% -----------------------------------------------------------------------
\section{Exercises}

% -------
\normalframetitle
\begin{frame}
\frametitle{A Faulty Reduction}

\begin{itemize}
\item \textbf{Rudrata path:} find a path that goes through each vertex exactly once.
  \begin{itemize}
  \item This is also known as the \textit{Hamiltonian path} problem.
  \end{itemize}
\item \textbf{Longest path:} find a simple path of length $\geq g$ [search formulation].
  \begin{itemize}
  \item \textit{Simple}: cannot pass through any vertex more than once.
  \end{itemize}
\end{itemize}

\begin{figure}
\includegraphics[width=0.4\textwidth, right]{hamiltonian}
\end{figure}

\end{frame}

% -------
\begin{frame}
\frametitle{A Faulty Reduction}
\begin{columns}[T]

% col1
\begin{column}{0.48\textwidth}
\small
\textbf{Undirected\texttt{ RUDRATA PATH }can be reduced to\texttt{ LONGEST PATH }in a DAG.}
\newline

Given a graph $G = (V, E)$, we can create a DAG as a directed DFS tree. If the longest path in this DAG has $|V| - 1$ edges, then there is a Rudrata path in $G$ (since a simple path with $|V| - 1$ edges visits every vertex).
\end{column}

% col2
\begin{column}{0.48\textwidth}
\small
\begin{itemize}
\item What is wrong with the given justification for our reduction?
\end{itemize}
\end{column}

\end{columns}
\end{frame}

% -------
\begin{frame}
\frametitle{A Faulty Reduction {\color{red} Solution}}
\begin{columns}[T]

% col1
\begin{column}{0.48\textwidth}
\small
\textbf{Undirected\texttt{ RUDRATA PATH }can be reduced to\texttt{ LONGEST PATH }in a DAG.}
\newline

Given a graph $G = (V, E)$, we can create a DAG as a directed DFS tree. If the longest path in this DAG has $|V| - 1$ edges, then there is a Rudrata path in $G$ (since a simple path with $|V| - 1$ edges visits every vertex).
\end{column}

% col2
\begin{column}{0.48\textwidth}
\small
\begin{itemize}
\item What is wrong with the given justification for our reduction?

{
  \color{red}
  To fully justify a reduction, we need to prove that an original problem instance $I$ has a solution \textbf{iff} reduced problem instance $I'$ has a solution.
  \begin{itemize}
  \item {\color{red} It is possible to produce a DAG without a length $|V| - 1$ path in cases where $G$ \textit{does} have a Rudrata path.}   
  \end{itemize}
}
\end{itemize}
\end{column}

\end{columns}
\end{frame}

\end{document}
