\documentclass[10pt]{beamer}

\usetheme[progressbar=frametitle]{metropolis}
\usepackage{appendixnumberbeamer}

\usepackage{booktabs}
\usepackage[scale=2]{ccicons}
\usepackage[export]{adjustbox}

\usepackage{pgfplots}
\usepgfplotslibrary{dateplot}

\usepackage{xspace}
\graphicspath{{figures/}}

\title{CS 170 Section 11}
\subtitle{Approximation Algorithms}
\date{April 11, 2018}
\author{Owen Jow}
\institute{University of California, Berkeley}

\begin{document}

\maketitle

\begin{frame}{Table of Contents}
  \setbeamertemplate{section in toc}[sections numbered]
  \tableofcontents[hideallsubsections]
\end{frame}

% -----------------------------------------------------------------------
\section{Reduction Review}

% -------
\begin{frame}[fragile]{Dominating Set}

\begin{figure}
  \includegraphics[width=0.5\textwidth]{domination}
  \caption{\textbf{dominating set.} A subset of vertices that either includes or touches (via an edge) every vertex in the graph.}
\end{figure}

\end{frame}

% -------
\begin{frame}[fragile]{Minimal Dominating Set}

\begin{columns}[T, onlytextwidth]
  \column{0.6\textwidth}
  \begin{block}{Minimal dominating set}
    A dominating set with $\leq k$ vertices.
  \end{block}

  Let $k = 2$:

  \begin{alertblock}{(a)}
    Not a minimal dominating set.
  \end{alertblock}

  \begin{exampleblock}{(b), (c)}
    Minimal dominating sets.
  \end{exampleblock}

  \column{0.4\textwidth}
  \includegraphics[width=0.9\textwidth]{mds}
\end{columns}

\end{frame}

% -------
\begin{frame}[fragile]{Proving NP-Hardness}

To prove that something is NP-hard, reduce a known NP-complete problem to it.

Recall:
\begin{itemize}
\item \textbf{NP-hard:} at least as hard as the NP-complete problems.
\item Difficulty flows in the direction of the reduction. If we reduce $A$ to $B$, then $B$ is at least as hard as $A$.
\end{itemize}

\end{frame}

% -------
\begin{frame}[fragile]{Some NP-Complete Problems}

\begin{columns}[T, onlytextwidth]
  \column{0.4\textwidth}
  \begin{table}
    \begin{tabular}{lr}
    \toprule
    Vertex cover & Subset sum \\
    Set cover & Longest path \\
    ZOE & Rudrata cycle \\
    MAX-2SAT & Dominating set \\
    SAT & Independent set \\
    Battleship & 3D matching \\
    Knapsack & Balanced cut \\
    Clique & Verbal arithmetic \\
    TSP & Optimal Rubik's cube \\
    ILP & Steiner tree (decision) \\
    \bottomrule
    \end{tabular}
  \end{table}

  \column{0.6\textwidth}
  \includegraphics[width=0.707\textwidth, right]{pneqnp}
\end{columns}

\end{frame}

% -------
\begin{frame}[fragile]{Exercise 1}

Argue that the \textit{minimal dominating set} problem is NP-hard.

\end{frame}

% -------
\begin{frame}[fragile]{Exercise 1 {\color{red} Solution}}

Argue that the \textit{minimal dominating set} problem is NP-hard.

{\color{red} We can reduce \textit{minimal set cover} to minimal dominating set.}

\end{frame}

% -----------------------------------------------------------------------
\section{Randomization for Approximation}

% -------
\begin{frame}[fragile]{Exposition}

NP-complete (and NP-hard) problems are everywhere. They're the shadows in the evening, the corruption in the government, the flyer people on Sproul.

What can be done? Assuming P $\neq$ NP, an optimal solution cannot be found in polynomial time.

So we must rely on alternatives. Intelligent exponential search is one of these. \textit{Approximation algorithms} are another.

\end{frame}

% -------
\begin{frame}[fragile]{Approximation Algorithms}

An \textbf{approximation algorithm} finds a solution with some guarantee of closeness to the optimum. Notably, it is \textit{efficient} (of polynomial time).

There are many ways to approximate (think of all the efficient problem-solving strategies you've learned so far!). Greedy and randomized approaches are popular, as they tend to be easy to formulate.

\end{frame}

% -------
\begin{frame}[fragile]{Exercise 2a}

Devise a randomized approximation algorithm for MAX-3SAT. It should achieve an approximation factor of $\frac{7}{8}$ in expectation.

Feel free to assume that each clause contains three distinct variables.

\end{frame}

% -------
\begin{frame}[fragile]{Exercise 2a {\color{red} Solution}}

{\color{red}
Randomly assign each variable a value. Let $X_i$ (for $i = 1, ..., n$) be a random variable that is $1$ if clause $i$ is satisfied and $0$ otherwise. Then
$$\mathbb{E}[X_i] = \left(0\right)\left(\frac{1}{8}\right) + \left(1\right)\left(\frac{7}{8}\right) = \frac{7}{8}$$
Let $X = \sum_{i = 1}^n X_i$ be the total number of clauses that are satisfied.
$$\mathbb{E}[X] = \mathbb{E}\left[\sum_{i = 1}^n X_i\right] = \sum_{i = 1}^n \mathbb{E}[X_i] = \sum_{i = 1}^n \frac{7}{8} = \frac{7}{8}n$$
Let $d^*$ be the optimal number of satisfied clauses. We have that $n \geq d^*$. Therefore, a random assignment is expected to satisfy $\frac{7}{8}n \geq \frac{7}{8}d^*$ clauses.
}

\end{frame}

% -------
\begin{frame}[fragile]{Exercise 2b}

The fact that $\mathbb{E}[X] = \frac{7}{8}n$ tells us something about every instance of MAX-3SAT.

Namely...?

\end{frame}

% -------
\begin{frame}[fragile]{Exercise 2b {\color{red} Solution}}

The fact that $\mathbb{E}[X] = \frac{7}{8}n$ tells us something about every instance of MAX-3SAT.

Namely...?

{\color{red} There always exists an assignment for which at least $\frac{7}{8}$ of all clauses are satisfied. Otherwise the expectation could not be $\frac{7}{8}$ of all clauses.}

\end{frame}

% -----------------------------------------------------------------------
\section{Fermat's Little Theorem as a Primality Test}

% -------
\begin{frame}{Fermat's Little Theorem}

\textbf{Fermat's little theorem:}

If $p$ is prime and $a$ is coprime with $p$, then $a^{p - 1} \equiv 1$ (mod $p$).

\begin{columns}[T, onlytextwidth]
  \column{0.5\textwidth}
  \column{0.5\textwidth}
  \metroset{block=fill}
  \begin{block}{$a, b$ coprime}
    The GCD of $a$ and $b$ is $1$.
  \end{block}
\end{columns}

\end{frame}

% -------
\begin{frame}{Fermat's Little Theorem as a Primality Test}

Say we want to determine whether $n$ is prime. We might think to use FLT as a primality test, i.e.
\begin{itemize}
\item Pick an arbitrary $a \in [1, n - 1]$ and compute $a^{n - 1}$ (mod $n$).
\item If this is equal to $1$, declare $n$ prime. Else declare $n$ composite.
\end{itemize}

\textit{But does this really work?} Spoilers: no.

\end{frame}

% -------
\begin{frame}[fragile]{Exercise 3a}

\begin{enumerate}[(i)]
\item Find an $a$ that will trick us into thinking that $15$ is prime.
\item Find an $a$ that will correctly identify $15$ as composite.
\end{enumerate}

\end{frame}

% -------
\begin{frame}[fragile]{Exercise 3a {\color{red} Solution}}

\begin{enumerate}[(i)]
\item Find an $a$ that will trick us into thinking that $15$ is prime. \\
{\color{red} $4$ will work for this. Note: when $n$ is composite but \\
$a^{n - 1} \equiv 1$ (mod $n$), we call $n$ a \textit{Fermat pseudoprime} to base $a$.}
\item Find an $a$ that will correctly identify $15$ as composite. {\color{red} 7.}
\end{enumerate}

\end{frame}

% -------
\begin{frame}[fragile]{Exercise 3b}

By FLT, primes will always be identified.

The problem is \textit{false positives} -- composite $n$ that masquerade as primes. There's one silver lining, though: if $a^{n - 1} \not\equiv 1$ (mod $n$) for some $a$ coprime to $n$, then \textbf{this must hold for at least half of the possible values of $a$}.

\end{frame}

% -------
\begin{frame}[fragile]{Exercise 3b}

Suppose there exists some $a$ in (mod $n$) s.t. $a^{n - 1} \not\equiv 1$ (mod $n$), where \\
$a$ is coprime with $n$. Show that $n$ is \textbf{not} Fermat-pseudoprime to at least half of the numbers in (mod $n$).

How can we use this to make our algorithm more effective?

\end{frame}

% -------
\begin{frame}[fragile]{Exercise 3b {\color{red} Solution}}

{\color{red}
For every $b$ s.t. $b^{n - 1} \equiv 1$ (mod $n$),
$$(ab)^{n - 1} = a^{n - 1}b^{n - 1} \not\equiv 1 \text{ (mod }n\text{)}$$
Since $a$ and $n$ are coprime, $a$ has an inverse modulo $n$. \\
Thus $ab$ is unique for every unique choice of $b$
($ab_1 \not\equiv ab_2$ iff $b_1 \not\equiv b_2$). \\
By extension, for every $b$ to which $n$ \textbf{is} Fermat-pseudoprime,
there is a unique $ab$ to which $n$ is \textbf{not} Fermat-pseudoprime.

\textit{We can make our algorithm more effective by checking a bunch of $a$ (not just one). The chance of being wrong $k$ times in a row is at most $\frac{1}{2^k}$.}
}

\end{frame}

% -------
\begin{frame}[fragile]{Exercise 3c}

Even with the improvement from (b), why might our algorithm still fail to be a good primality test?

\end{frame}

% -------
\begin{frame}[fragile]{Exercise 3c {\color{red} Solution}}

Even with the improvement from (b), why might our algorithm still fail to be a good primality test?

{\color{red}
In order to correctly identify composite $n$, we need an $a$ coprime with $n$ s.t. $a^{n - 1} \not\equiv 1$ (mod $n$). But there is no guarantee that such an $a$ exists!

(Such composite $n$, which pass the FLT primality test for all $a$, are called \textit{Carmichael numbers}.)
}

\end{frame}

\end{document}
