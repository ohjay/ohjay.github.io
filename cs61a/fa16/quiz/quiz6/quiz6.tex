\documentclass[twoside]{article}
\usepackage{../quiz}
\usepackage{forest}

\pagestyle{myheadings}

\lstset{
    language=Python,
    basicstyle=\ttfamily,
    showstringspaces=false
    keywordstyle=\color{black},
    commentstyle=\color{black},
    stringstyle=\color{black},
}

\def\semester{Fall 2016}

%%% Showing solutions %%%

\def\doshow{1}
\ifx\doshow\showsolution
\newcommand{\solution}[1]{{\color{red}#1}}
\newcommand{\solutioncircle}[1]{{\color{red}#1}}
\newcommand{\solutionimage}[2]{#2} % first arg is question, second is solution
\newcommand{\solutionblank}[2]{\hbox to #2{\color{red}#1}}
\else
\newcommand\solution[1]{} % excludes
\newcommand{\solutioncircle}[1]{#1} % don't color text but still display it
\newcommand{\solutionimage}[2]{#1} % first arg is question, second is solution
\newcommand{\solutionblank}[2]{{\rule{0pt}{2em}\underline{\hbox to #2{}}}}
\fi
\usepackage{multicol}

%%% Actual content begins here %%%
\title{\sc Midterm 2}

\begin{document}
\thispagestyle{empty}
\maketitle

\leavevmode\\
\textbf{Post-MT1 topics that you'll want to have down cold (not necessarily a comprehensive list, but you better know all about these!)}: nonlocality, OOP, data abstraction, trees, linked lists, order of growth, mutation

\begin{enumerate}
%%% Q1: Linked List Basics %%%
\q{4}{Linked List Basics}

For each of the following code fragments, add arrows and values to the object skeletons to the right to show the final state of the program.  Single boxes are variables that contain pointers.  Double boxes are \lstinline/Links/. Not all boxes will necessarily be used. (Note: I suggest doing it on scratch paper before filling in the boxes.)

\vspace{0.2in}

\solutionimage{\begin{minipage}{2.65in}
\begin{verbatim}
C = Link(1, Link(6))
C.rest.rest = Link(C.first, \
        Link(C.rest.first))
S = C.rest.rest
C.rest.rest = C.rest.rest.rest
C.rest.rest.first = C.rest
C, C.rest.rest = S.rest.first, C
\end{verbatim}
\end{minipage}
\begin{minipage}{5in}
\expandafter\ifx\csname graph\endcsname\relax
   \csname newbox\expandafter\endcsname\csname graph\endcsname
\fi
\ifx\graphtemp\undefined
  \csname newdimen\endcsname\graphtemp
\fi
\expandafter\setbox\csname graph\endcsname
 =\vtop{\vskip 0pt\hbox{%
    \special{pn 8}%
    \special{pa 0 300}%
    \special{pa 300 300}%
    \special{pa 300 0}%
    \special{pa 0 0}%
    \special{pa 0 300}%
    \special{fp}%
    \graphtemp=.5ex
    \advance\graphtemp by 0.150in
    \rlap{\kern 0.000in\lower\graphtemp\hbox to 0pt{\hss C:~~}}%
    \special{pa 0 750}%
    \special{pa 300 750}%
    \special{pa 300 450}%
    \special{pa 0 450}%
    \special{pa 0 750}%
    \special{fp}%
    \graphtemp=.5ex
    \advance\graphtemp by 0.600in
    \rlap{\kern 0.000in\lower\graphtemp\hbox to 0pt{\hss S:~~}}%
    \special{pa 600 300}%
    \special{pa 900 300}%
    \special{pa 900 0}%
    \special{pa 600 0}%
    \special{pa 600 300}%
    \special{fp}%
    \special{pa 900 300}%
    \special{pa 1200 300}%
    \special{pa 1200 0}%
    \special{pa 900 0}%
    \special{pa 900 300}%
    \special{fp}%
    \special{pa 1500 300}%
    \special{pa 1800 300}%
    \special{pa 1800 0}%
    \special{pa 1500 0}%
    \special{pa 1500 300}%
    \special{fp}%
    \special{pa 1800 300}%
    \special{pa 2100 300}%
    \special{pa 2100 0}%
    \special{pa 1800 0}%
    \special{pa 1800 300}%
    \special{fp}%
    \special{pa 2400 300}%
    \special{pa 2700 300}%
    \special{pa 2700 0}%
    \special{pa 2400 0}%
    \special{pa 2400 300}%
    \special{fp}%
    \special{pa 2700 300}%
    \special{pa 3000 300}%
    \special{pa 3000 0}%
    \special{pa 2700 0}%
    \special{pa 2700 300}%
    \special{fp}%
    \special{pa 3300 300}%
    \special{pa 3600 300}%
    \special{pa 3600 0}%
    \special{pa 3300 0}%
    \special{pa 3300 300}%
    \special{fp}%
    \special{pa 3600 300}%
    \special{pa 3900 300}%
    \special{pa 3900 0}%
    \special{pa 3600 0}%
    \special{pa 3600 300}%
    \special{fp}%
    \special{pa 600 750}%
    \special{pa 900 750}%
    \special{pa 900 450}%
    \special{pa 600 450}%
    \special{pa 600 750}%
    \special{fp}%
    \special{pa 900 750}%
    \special{pa 1200 750}%
    \special{pa 1200 450}%
    \special{pa 900 450}%
    \special{pa 900 750}%
    \special{fp}%
    \special{pa 1500 750}%
    \special{pa 1800 750}%
    \special{pa 1800 450}%
    \special{pa 1500 450}%
    \special{pa 1500 750}%
    \special{fp}%
    \special{pa 1800 750}%
    \special{pa 2100 750}%
    \special{pa 2100 450}%
    \special{pa 1800 450}%
    \special{pa 1800 750}%
    \special{fp}%
    \special{pa 2400 750}%
    \special{pa 2700 750}%
    \special{pa 2700 450}%
    \special{pa 2400 450}%
    \special{pa 2400 750}%
    \special{fp}%
    \special{pa 2700 750}%
    \special{pa 3000 750}%
    \special{pa 3000 450}%
    \special{pa 2700 450}%
    \special{pa 2700 750}%
    \special{fp}%
    \special{pa 3300 750}%
    \special{pa 3600 750}%
    \special{pa 3600 450}%
    \special{pa 3300 450}%
    \special{pa 3300 750}%
    \special{fp}%
    \special{pa 3600 750}%
    \special{pa 3900 750}%
    \special{pa 3900 450}%
    \special{pa 3600 450}%
    \special{pa 3600 750}%
    \special{fp}%
    \hbox{\vrule depth0.750in width0pt height 0pt}%
    \kern 3.900in
  }%
}%
\box\graph
\end{minipage}
}{\begin{minipage}{2.65in}
\begin{verbatim}
C = Link(1, Link(6))
C.rest.rest = Link(C.first, \
        Link(C.rest.first))
S = C.rest.rest
C.rest.rest = C.rest.rest.rest
C.rest.rest.first = C.rest
C, C.rest.rest = S.rest.first, C
\end{verbatim}
\end{minipage}
\begin{minipage}{5in}
\expandafter\ifx\csname graph\endcsname\relax
   \csname newbox\expandafter\endcsname\csname graph\endcsname
\fi
\ifx\graphtemp\undefined
  \csname newdimen\endcsname\graphtemp
\fi
\expandafter\setbox\csname graph\endcsname
 =\vtop{\vskip 0pt\hbox{%
    \special{pn 8}%
    \special{pa 0 600}%
    \special{pa 300 600}%
    \special{pa 300 300}%
    \special{pa 0 300}%
    \special{pa 0 600}%
    \special{fp}%
    \graphtemp=.5ex
    \advance\graphtemp by 0.450in
    \rlap{\kern 0.000in\lower\graphtemp\hbox to 0pt{\hss C:~~}}%
    \special{pa 0 1050}%
    \special{pa 300 1050}%
    \special{pa 300 750}%
    \special{pa 0 750}%
    \special{pa 0 1050}%
    \special{fp}%
    \graphtemp=.5ex
    \advance\graphtemp by 0.900in
    \rlap{\kern 0.000in\lower\graphtemp\hbox to 0pt{\hss S:~~}}%
    \special{pa 600 600}%
    \special{pa 900 600}%
    \special{pa 900 300}%
    \special{pa 600 300}%
    \special{pa 600 600}%
    \special{fp}%
    \graphtemp=.5ex
    \advance\graphtemp by 0.450in
    \rlap{\kern 0.750in\lower\graphtemp\hbox to 0pt{\hss 6\hss}}%
    \special{pa 900 600}%
    \special{pa 1200 600}%
    \special{pa 1200 300}%
    \special{pa 900 300}%
    \special{pa 900 600}%
    \special{fp}%
    \special{pa 1500 600}%
    \special{pa 1800 600}%
    \special{pa 1800 300}%
    \special{pa 1500 300}%
    \special{pa 1500 600}%
    \special{fp}%
    \graphtemp=.5ex
    \advance\graphtemp by 0.450in
    \rlap{\kern 1.650in\lower\graphtemp\hbox to 0pt{\hss 1\hss}}%
    \special{pa 1800 600}%
    \special{pa 2100 600}%
    \special{pa 2100 300}%
    \special{pa 1800 300}%
    \special{pa 1800 600}%
    \special{fp}%
    \special{pa 2400 600}%
    \special{pa 2700 600}%
    \special{pa 2700 300}%
    \special{pa 2400 300}%
    \special{pa 2400 600}%
    \special{fp}%
    \graphtemp=.5ex
    \advance\graphtemp by 0.450in
    \rlap{\kern 2.550in\lower\graphtemp\hbox to 0pt{\hss 1\hss}}%
    \special{pa 2700 600}%
    \special{pa 3000 600}%
    \special{pa 3000 300}%
    \special{pa 2700 300}%
    \special{pa 2700 600}%
    \special{fp}%
    \special{pa 3300 600}%
    \special{pa 3600 600}%
    \special{pa 3600 300}%
    \special{pa 3300 300}%
    \special{pa 3300 600}%
    \special{fp}%
    \special{pa 3600 600}%
    \special{pa 3900 600}%
    \special{pa 3900 300}%
    \special{pa 3600 300}%
    \special{pa 3600 600}%
    \special{fp}%
    \special{pa 600 1050}%
    \special{pa 900 1050}%
    \special{pa 900 750}%
    \special{pa 600 750}%
    \special{pa 600 1050}%
    \special{fp}%
    \special{pa 900 1050}%
    \special{pa 1200 1050}%
    \special{pa 1200 750}%
    \special{pa 900 750}%
    \special{pa 900 1050}%
    \special{fp}%
    \special{pa 1500 1050}%
    \special{pa 1800 1050}%
    \special{pa 1800 750}%
    \special{pa 1500 750}%
    \special{pa 1500 1050}%
    \special{fp}%
    \special{pa 1800 1050}%
    \special{pa 2100 1050}%
    \special{pa 2100 750}%
    \special{pa 1800 750}%
    \special{pa 1800 1050}%
    \special{fp}%
    \special{pa 2400 1050}%
    \special{pa 2700 1050}%
    \special{pa 2700 750}%
    \special{pa 2400 750}%
    \special{pa 2400 1050}%
    \special{fp}%
    \special{pa 2700 1050}%
    \special{pa 3000 1050}%
    \special{pa 3000 750}%
    \special{pa 2700 750}%
    \special{pa 2700 1050}%
    \special{fp}%
    \special{pa 3300 1050}%
    \special{pa 3600 1050}%
    \special{pa 3600 750}%
    \special{pa 3300 750}%
    \special{pa 3300 1050}%
    \special{fp}%
    \special{pa 3600 1050}%
    \special{pa 3900 1050}%
    \special{pa 3900 750}%
    \special{pa 3600 750}%
    \special{pa 3600 1050}%
    \special{fp}%
    \special{sh 1.000}%
    \special{pn 1}%
    \special{pa 500 425}%
    \special{pa 600 450}%
    \special{pa 500 475}%
    \special{pa 500 425}%
    \special{fp}%
    \special{pn 14}%
    \special{pa 150 450}%
    \special{pa 500 450}%
    \special{fp}%
    \special{sh 1.000}%
    \special{pn 1}%
    \special{pa 3286 348}%
    \special{pa 3300 450}%
    \special{pa 3240 366}%
    \special{pa 3286 348}%
    \special{fp}%
    \special{pn 14}%
    \special{pa 1050 450}%
    \special{pa 1900 200}%
    \special{pa 3200 200}%
    \special{pa 3263 357}%
    \special{fp}%
    \special{sh 1.000}%
    \special{pn 1}%
    \special{pa 586 348}%
    \special{pa 600 450}%
    \special{pa 540 366}%
    \special{pa 586 348}%
    \special{fp}%
    \special{pn 14}%
    \special{pa 1950 450}%
    \special{pa 1700 200}%
    \special{pa 500 200}%
    \special{pa 563 357}%
    \special{fp}%
    \special{sh 1.000}%
    \special{pn 1}%
    \special{pa 2297 445}%
    \special{pa 2400 450}%
    \special{pa 2307 494}%
    \special{pa 2297 445}%
    \special{fp}%
    \special{pn 14}%
    \special{pa 150 900}%
    \special{pa 2302 470}%
    \special{fp}%
    \special{sh 1.000}%
    \special{pn 1}%
    \special{pa 3200 425}%
    \special{pa 3300 450}%
    \special{pa 3200 475}%
    \special{pa 3200 425}%
    \special{fp}%
    \special{pn 14}%
    \special{pa 2850 450}%
    \special{pa 3200 450}%
    \special{fp}%
    \special{sh 1.000}%
    \special{pn 1}%
    \special{pa 540 534}%
    \special{pa 600 450}%
    \special{pa 586 552}%
    \special{pa 540 534}%
    \special{fp}%
    \special{pn 14}%
    \special{pa 3450 450}%
    \special{pa 3200 700}%
    \special{pa 500 700}%
    \special{pa 563 543}%
    \special{fp}%
    \special{sh 1.000}%
    \special{pn 1}%
    \special{pa 1482 348}%
    \special{pa 1500 450}%
    \special{pa 1437 369}%
    \special{pa 1482 348}%
    \special{fp}%
    \special{pn 14}%
    \special{pa 3750 450}%
    \special{pa 3500 0}%
    \special{pa 1300 0}%
    \special{pa 1459 359}%
    \special{fp}%
    \hbox{\vrule depth1.050in width0pt height 0pt}%
    \kern 3.900in
  }%
}%
\box\graph
\end{minipage}
}

\vspace{0.2in}

Fill in the code to create the linked lists below. No multiple assignment allowed. Incidentally, as a general rule (i.e. in real life) try not to set \texttt{first} elements to other linked lists.

\vspace{0.2in}

\begin{minipage}{5in}
\expandafter\ifx\csname graph\endcsname\relax
   \csname newbox\expandafter\endcsname\csname graph\endcsname
\fi
\ifx\graphtemp\undefined
  \csname newdimen\endcsname\graphtemp
\fi
\expandafter\setbox\csname graph\endcsname
 =\vtop{\vskip 0pt\hbox{%
    \special{pn 8}%
    \special{pa 300 300}%
    \special{pa 600 300}%
    \special{pa 600 0}%
    \special{pa 300 0}%
    \special{pa 300 300}%
    \special{fp}%
    \graphtemp=.5ex
    \advance\graphtemp by 0.150in
    \rlap{\kern 0.300in\lower\graphtemp\hbox to 0pt{\hss A:~~}}%
    \special{pa 300 750}%
    \special{pa 600 750}%
    \special{pa 600 450}%
    \special{pa 300 450}%
    \special{pa 300 750}%
    \special{fp}%
    \graphtemp=.5ex
    \advance\graphtemp by 0.600in
    \rlap{\kern 0.300in\lower\graphtemp\hbox to 0pt{\hss B:~~}}%
    \special{pa 900 300}%
    \special{pa 1200 300}%
    \special{pa 1200 0}%
    \special{pa 900 0}%
    \special{pa 900 300}%
    \special{fp}%
    \graphtemp=.5ex
    \advance\graphtemp by 0.150in
    \rlap{\kern 1.050in\lower\graphtemp\hbox to 0pt{\hss 0\hss}}%
    \special{pa 1200 300}%
    \special{pa 1500 300}%
    \special{pa 1500 0}%
    \special{pa 1200 0}%
    \special{pa 1200 300}%
    \special{fp}%
    \special{pa 1800 300}%
    \special{pa 2100 300}%
    \special{pa 2100 0}%
    \special{pa 1800 0}%
    \special{pa 1800 300}%
    \special{fp}%
    \graphtemp=.5ex
    \advance\graphtemp by 0.150in
    \rlap{\kern 1.950in\lower\graphtemp\hbox to 0pt{\hss 1\hss}}%
    \special{pa 2100 300}%
    \special{pa 2400 300}%
    \special{pa 2400 0}%
    \special{pa 2100 0}%
    \special{pa 2100 300}%
    \special{fp}%
    \special{pa 2700 300}%
    \special{pa 3000 300}%
    \special{pa 3000 0}%
    \special{pa 2700 0}%
    \special{pa 2700 300}%
    \special{fp}%
    \graphtemp=.5ex
    \advance\graphtemp by 0.150in
    \rlap{\kern 2.850in\lower\graphtemp\hbox to 0pt{\hss 2\hss}}%
    \special{pa 3000 300}%
    \special{pa 3300 300}%
    \special{pa 3300 0}%
    \special{pa 3000 0}%
    \special{pa 3000 300}%
    \special{fp}%
    \special{pa 3600 300}%
    \special{pa 3900 300}%
    \special{pa 3900 0}%
    \special{pa 3600 0}%
    \special{pa 3600 300}%
    \special{fp}%
    \graphtemp=.5ex
    \advance\graphtemp by 0.150in
    \rlap{\kern 3.750in\lower\graphtemp\hbox to 0pt{\hss 3\hss}}%
    \special{pa 3900 300}%
    \special{pa 4200 300}%
    \special{pa 4200 0}%
    \special{pa 3900 0}%
    \special{pa 3900 300}%
    \special{fp}%
    \special{pa 900 750}%
    \special{pa 1200 750}%
    \special{pa 1200 450}%
    \special{pa 900 450}%
    \special{pa 900 750}%
    \special{fp}%
    \special{pa 1200 750}%
    \special{pa 1500 750}%
    \special{pa 1500 450}%
    \special{pa 1200 450}%
    \special{pa 1200 750}%
    \special{fp}%
    \special{pa 1800 750}%
    \special{pa 2100 750}%
    \special{pa 2100 450}%
    \special{pa 1800 450}%
    \special{pa 1800 750}%
    \special{fp}%
    \graphtemp=.5ex
    \advance\graphtemp by 0.600in
    \rlap{\kern 1.950in\lower\graphtemp\hbox to 0pt{\hss 4\hss}}%
    \special{pa 2100 750}%
    \special{pa 2400 750}%
    \special{pa 2400 450}%
    \special{pa 2100 450}%
    \special{pa 2100 750}%
    \special{fp}%
    \special{pa 2700 750}%
    \special{pa 3000 750}%
    \special{pa 3000 450}%
    \special{pa 2700 450}%
    \special{pa 2700 750}%
    \special{fp}%
    \graphtemp=.5ex
    \advance\graphtemp by 0.600in
    \rlap{\kern 2.850in\lower\graphtemp\hbox to 0pt{\hss 5\hss}}%
    \special{pa 3000 750}%
    \special{pa 3300 750}%
    \special{pa 3300 450}%
    \special{pa 3000 450}%
    \special{pa 3000 750}%
    \special{fp}%
    \special{pa 3600 750}%
    \special{pa 3900 750}%
    \special{pa 3900 450}%
    \special{pa 3600 450}%
    \special{pa 3600 750}%
    \special{fp}%
    \graphtemp=.5ex
    \advance\graphtemp by 0.600in
    \rlap{\kern 3.750in\lower\graphtemp\hbox to 0pt{\hss 6\hss}}%
    \special{pa 3900 750}%
    \special{pa 4200 750}%
    \special{pa 4200 450}%
    \special{pa 3900 450}%
    \special{pa 3900 750}%
    \special{fp}%
    \special{sh 1.000}%
    \special{pn 1}%
    \special{pa 800 125}%
    \special{pa 900 150}%
    \special{pa 800 175}%
    \special{pa 800 125}%
    \special{fp}%
    \special{pn 14}%
    \special{pa 450 150}%
    \special{pa 800 150}%
    \special{fp}%
    \special{sh 1.000}%
    \special{pn 1}%
    \special{pa 1700 125}%
    \special{pa 1800 150}%
    \special{pa 1700 175}%
    \special{pa 1700 125}%
    \special{fp}%
    \special{pn 14}%
    \special{pa 1350 150}%
    \special{pa 1700 150}%
    \special{fp}%
    \special{sh 1.000}%
    \special{pn 1}%
    \special{pa 2600 125}%
    \special{pa 2700 150}%
    \special{pa 2600 175}%
    \special{pa 2600 125}%
    \special{fp}%
    \special{pn 14}%
    \special{pa 2250 150}%
    \special{pa 2600 150}%
    \special{fp}%
    \special{sh 1.000}%
    \special{pn 1}%
    \special{pa 3500 125}%
    \special{pa 3600 150}%
    \special{pa 3500 175}%
    \special{pa 3500 125}%
    \special{fp}%
    \special{pn 14}%
    \special{pa 3150 150}%
    \special{pa 3500 150}%
    \special{fp}%
    \special{sh 1.000}%
    \special{pn 1}%
    \special{pa 3547 512}%
    \special{pa 3600 600}%
    \special{pa 3512 547}%
    \special{pa 3547 512}%
    \special{fp}%
    \special{pn 14}%
    \special{pa 4050 150}%
    \special{pa 3450 450}%
    \special{pa 3529 529}%
    \special{fp}%
    \special{sh 1.000}%
    \special{pn 1}%
    \special{pa 800 575}%
    \special{pa 900 600}%
    \special{pa 800 625}%
    \special{pa 800 575}%
    \special{fp}%
    \special{pn 14}%
    \special{pa 450 600}%
    \special{pa 800 600}%
    \special{fp}%
    \special{sh 1.000}%
    \special{pn 1}%
    \special{pa 833 228}%
    \special{pa 900 150}%
    \special{pa 878 251}%
    \special{pa 833 228}%
    \special{fp}%
    \special{pn 14}%
    \special{pa 1050 600}%
    \special{pa 750 450}%
    \special{pa 855 239}%
    \special{fp}%
    \special{sh 1.000}%
    \special{pn 1}%
    \special{pa 1700 575}%
    \special{pa 1800 600}%
    \special{pa 1700 625}%
    \special{pa 1700 575}%
    \special{fp}%
    \special{pn 14}%
    \special{pa 1350 600}%
    \special{pa 1700 600}%
    \special{fp}%
    \special{sh 1.000}%
    \special{pn 1}%
    \special{pa 3533 678}%
    \special{pa 3600 600}%
    \special{pa 3578 701}%
    \special{pa 3533 678}%
    \special{fp}%
    \special{pn 14}%
    \special{pa 2250 600}%
    \special{pa 2550 900}%
    \special{pa 3450 900}%
    \special{pa 3555 689}%
    \special{fp}%
    \special{sh 1.000}%
    \special{pn 1}%
    \special{pa 3500 575}%
    \special{pa 3600 600}%
    \special{pa 3500 625}%
    \special{pa 3500 575}%
    \special{fp}%
    \special{pn 14}%
    \special{pa 3150 600}%
    \special{pa 3500 600}%
    \special{fp}%
    \special{sh 1.000}%
    \special{pn 1}%
    \special{pa 2628 674}%
    \special{pa 2700 600}%
    \special{pa 2671 699}%
    \special{pa 2628 674}%
    \special{fp}%
    \special{pn 14}%
    \special{pa 4050 600}%
    \special{pa 4050 960}%
    \special{pa 2490 960}%
    \special{pa 2650 686}%
    \special{fp}%
    \hbox{\vrule depth0.960in width0pt height 0pt}%
    \kern 4.200in
  }%
}%
\box\graph
\end{minipage}


\vspace{0.2in}

\begin{lstlisting}
A = Link(0, Link(1, Link(2, Link(3))))

B = ______________________________________________________________________

B._______________________________ = ______________________________________

B._______________________________ = ______________________________________
\end{lstlisting}

%%% Q2: There Is No Trie %%%
\q{6}{There Is No Trie}

Say we had a tree that stored English words character-by-character. The first level of the tree would contain the first character of a word, the second level the second character... and so on so forth. At the end of a complete word in the tree, we would store the full word (using all the characters up to that point) as the node value. If the string of characters leading up to some node was NOT a word, we would store \lstinline/None/ as its value.

Let's make this tree a reality. To facilitate such an undertaking, we edit our \lstinline/Tree/ class so that branches are stored as a dictionary from characters to \lstinline/Trees/.

We include on the next page an example of a tree containing the strings {\tt\{body, be, boo, ate, ask, at, as, a\}}, along with the modified \lstinline/Tree/ definition. Here, edges are marked with the key in the \lstinline/branches/ dictionary that they correspond to.

\begin{minipage}{2.35in}
\begin{forest}
for tree={circle, draw, l sep=15pt}
[\lstinline/None/
    [\lstinline/`a'/, edge label={node[midway,left] {a}}
      [\lstinline/`at'/, edge label={node[midway,left] {t}}
        [\lstinline/`ate'/, edge label={node[midway,left] {e}}]
      ]
      [\lstinline/`as'/, edge label={node[midway,left] {s}}
        [\lstinline/`ask'/, edge label={node[midway,left] {k}}]
      ]
    ]
    [\lstinline/None/, edge label={node[midway,left] {b}}
      [\lstinline/None/, edge label={node[midway,left] {o}}
        [\lstinline/`boo'/, edge label={node[midway,left] {o}}]
        [\lstinline/None/, edge label={node[midway,left] {d}}
          [\lstinline/`body'/, edge label={node[midway,left] {y}}]
        ]
      ]
      [\lstinline/`be'/, edge label={node[midway,left] {e}}]
    ]
  ] 
]
\end{forest}
\end{minipage}
\begin{minipage}{5in}
\begin{lstlisting}
class Tree:
    def __init__(self, root=None, branches={}):
        self.root = root
        self.branches = branches
    
    def insert(self, chars, word):
        if len(chars) == 0:
            self.root = word
            return
        if chars[0] not in self.branches:
            self.branches[chars[0]] = Tree(None, {})
        self.branches[chars[0]].insert(chars[1:], \
                word)
    
    def autocomplete(self, prefix):
        ...
    
    def get_all_words(self):
        ...
\end{lstlisting}
\end{minipage}

\medskip

To create the tree on the left, for instance, you could execute {\tt tree = Tree(); [tree.insert(list(word), word) for word in [`ate', `ask', `boo', `body', `be', `a', `at', `as']]}. (Obligatory note: you normally shouldn't use list comprehensions for stuff like this. I just didn't want the quiz to be three pages.)

We can process our word tree in many interesting -- and efficient -- ways. One example is autocompletion. \textit{Fill in the blanks below so that} \lstinline/autocomplete/ \textit{(a method of our} \lstinline/Tree/ \textit{class!) returns a list of full words pertaining to the given prefix. For example,} \lstinline/tree.autocomplete(`bo')/ \textit{would return} \lstinline/[`boo', `body']/ \textit{if} \lstinline/tree/ \textit{were the tree from above. You will probably need to implement and use the} \lstinline/get_all_words/ \textit{method, which returns a list of all of the words in a tree.}

\lstinputlisting{autocomplete.py}


\end{enumerate}
\end{document}
