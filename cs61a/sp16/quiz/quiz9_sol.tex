\documentclass[twoside]{article}
\usepackage{quiz}

\pagestyle{myheadings}

\lstset{
    language=Python,
    basicstyle=\ttfamily,
    showstringspaces=false
    keywordstyle=\color{black},
    commentstyle=\color{black},
    stringstyle=\color{black},
}

\def\semester{Spring 2016}

\newcommand{\solution}[1]{{\color{red}#1}}
% \newcommand\solution[1]{}

%%% Actual content begins here %%%
\title{\sc Discussion Quiz 9 \solution{Solutions}}

\begin{document}
\thispagestyle{empty}
\maketitle

\begin{enumerate}
%%% Question 1: Chasing Tails %%%
\q{3}{Chasing Tails}

Identify whether or not each of the following procedures uses a constant amount of space in a tail-recursive Scheme implementation (i.e. whether \textbf{every} recursive call is a tail call).

%%% 1(a) %%%
\begin{lstlisting}
(define (copy lst result)
    (if (null? lst) result
        ((lambda (copy) copy) (copy (cdr lst)
                                    (append result (list (car lst)))))))
\end{lstlisting}

(Note: {\tt append} takes two or more lists and constructs a new list with all of their elements.)

\solution{{\tt copy} is \emph{not} tail-recursive. After the recursive call returns, we still have to apply a {\tt lambda} procedure.}
\newline

%%% 1(b) %%%
\begin{lstlisting}
(define (broken lst) (broken (broken lst)))
\end{lstlisting}

\solution{{\tt broken} is \emph{not} tail-recursive. One of the recursive calls is not a tail call.}
\newline

%%% 1(c) %%%
\begin{lstlisting}
(define (is-ascending lst last-num)
    (if (null? lst) #t
        (and (is-ascending (cdr lst) (car lst)) (> (car lst) last-num))))
\end{lstlisting}

(This subroutine would need to be called with a {\tt last-num} that is less than all of the elements in the list.)

\solution{{\tt is-ascending} is \emph{not} tail-recursive. The recursive call isn't even in a tail context!}
\newline

%%% Question 2: It's Hailing... Again %%%
\q{3}{It's Hailing... Again}

Write a \emph{tail-recursive} version of {\tt hailstone}. This procedure accepts a positive integer {\tt n} and an empty list {\tt lst}, and returns a list containing the hailstone sequence that starts at {\tt n}.

As an example, {\tt (hailstone 5 '())} would return {\tt (5 16 8 4 2 1)}.

\lstset{
    basicstyle=\ttfamily\color{red},
}

\begin{lstlisting}
(define (hailstone n lst)
  (cond ((= n 1) (append lst (list 1)))
        ((even? n) (hailstone (/ n 2) (append lst (list n))))
        (else (hailstone (+ 1 (* 3 n)) (append lst (list n))))))
\end{lstlisting}

%%% Question 3: Humans Need Not Apply %%%
\q{4}{Humans Need Not Apply}

What does {\tt eval} do (in the context of an interpreter)? What does {\tt apply} do?

\solution{{\tt eval} parses expressions (all kinds of expressions; {\tt eval} doesn't discriminate!), \emph{evaluating} an expression's form to determine its meaning. On the other hand, {\tt apply} handles function calls (it \emph{applies} operators to arguments).

{\tt eval} and {\tt apply} are mutually recursive. Whenever {\tt eval} encounters a function call, it sends the expression to {\tt apply} to do the actual calling. In turn, {\tt apply} uses {\tt eval} to evaluate its arguments and to parse the body of user-defined functions.}

\end{enumerate}
\end{document}
