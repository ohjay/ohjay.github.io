\documentclass[11pt]{article}

\usepackage{textcomp, listings, enumitem}
\usepackage[a4paper, total={6in, 10in}]{geometry}
\usepackage[english]{babel}

\lstset{
    basicstyle=\fontsize{10}{13}\ttfamily,
    numberstyle=\tiny,
    numbersep=5pt,
    tabsize=2,
    extendedchars=true,
    breaklines=true,
    aboveskip=5mm,
    belowskip=5mm,
    frame=tb,
    showspaces=false,
    showtabs=false,
    framextopmargin=4pt,
    framexbottommargin=4pt,
    showstringspaces=false,
    language=Lisp,
    upquote=true,
    escapeinside={\%*}{*)}
}

\oddsidemargin=0in
\evensidemargin=0in
\textwidth=6.3in

\parindent=0in
\pagestyle{empty}

\begin{document}
\textbf{61A-125 Spring 2016}\newline
\textbf{Discussion Quiz 08}\newline

%%% Q1: A Scheme Primer %%%
\textbf{1. A Scheme Primer}\newline
Please answer the following questions about Scheme.

\begin{enumerate}[leftmargin=0.63cm,itemindent=0cm,labelwidth=\itemindent,label=(\alph*)]
\item What is Scheme?
\newline\newline\newline

\item Do you enjoy counting parentheses? Circle one: Yes

\item In Scheme, how would I define the value of \texttt{a} to be 5?
\newline\newline\newline

\item Define a Scheme procedure, \texttt{remainder}, that takes in two integers \texttt{m} and \texttt{n} and returns the remainder of \texttt{m} divided by \texttt{n}. Assume that both \texttt{m} and \texttt{n} are positive.\newline
\end{enumerate}
\vspace{\fill}

%%% Q2: WWSP? %%%
\textbf{2. WWSP?}
\begin{lstlisting}
(+ (and 0 (or #f 1)) 2)


(cons 5 6)


(define x (+))
(define y +)
(x 3 4)


(y 3 4)


(define (mystery lst)
  (cond ((null? lst) nil)
    ((= (modulo (car lst) 2) 1) (cons (car lst) (mystery (cdr lst))))
    (else (mystery (cdr lst)))))
(mystery (list 1 2 3 4))
%*\newline*)
%*\newline*)
\end{lstlisting}
\end{document}
