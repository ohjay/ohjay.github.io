\documentclass[twoside]{article}
\usepackage{../quiz}
\usepackage{fancyhdr}

\pagestyle{fancy}
\renewcommand{\headrulewidth}{0pt}
\cfoot{}
\rfoot{Solutions to these quizzes will be posted at \textbf{\href{http://owenjow.xyz/cs61a/section-quizzes/}{owenjow.xyz/cs61a/section-quizzes}}.}
\renewcommand{\footrulewidth}{0.4pt}

\lstset{
    language=Python,
    basicstyle=\ttfamily,
    showstringspaces=false
    keywordstyle=\color{black},
    commentstyle=\color{black},
    stringstyle=\color{black},
    escapeinside={<*}{*>},
}

\def\semester{Spring 2k17}
\newcommand{\solution}[1]{{\color{red}#1}}

%%% Actual, flexible content begins here %%%
\title{\sc Midterm 1 is in 22 days}

\begin{document}
\maketitle

\begin{enumerate}
%%% Q1 %%%
\q{2}{Back to the Stone Age}

Are these primitive expressions or call expressions? (Answer for each one individually.)
\begin{enumerate}
\item \texttt{5}
\item \texttt{add(7, 7)}
\item \texttt{print(3)}
\item \texttt{4 + 5}
\end{enumerate}

%%% Q2 %%%
\q{3}{Control Yourself}

Imagine you have the boolean values \texttt{isRaining}, \texttt{fallsBehind}, \texttt{doesProcrastinate}, \texttt{numExamsTaken == 45}, and \texttt{addictedToCandy}. Using only these values and the operators \texttt{and}, \texttt{or}, and \texttt{not}, fill in the blank with a boolean expression that represents -- based on the following quote -- whether or not you will ace this class:

\begin{quote}
``I will ace this class if I keep up with the material, don't procrastinate, and take all the past exams$^*$. Even if I don't do that stuff, I will ace this class if it is raining. Conversely, I will fail this class if I eat too much candy."
\end{quote}

\hfill $^*$\textit{There are 45 past exams.}

\begin{lstlisting}
will_ace_this_class = \

________________________________________________________________________ \

__________________________________________________________________________
\end{lstlisting}

%%% Q3 %%%
\q{5}{Functions of a Higher Order}

Write a function that
\begin{enumerate}
\item takes a single function as input, and
\item returns a function that does the same thing as the input function, but also prints ``\texttt{gr8 m8 i r8 8/8}" every eighth time the function is called.
\end{enumerate}

To deal with variability in the number of arguments to the input function, use \texttt{*args} as both your inner function's formal parameters and the input function's arguments. (\texttt{*args} packs an arbitrary number of arguments into a tuple in the first case, and unpacks all of them into separate positional arguments in the second case.)

\textit{A function skeleton has been provided for you on the next page. Fill in your answer there.}

\newpage

\begin{lstlisting}
def hof8(___________________):
    """Returns a ver. of the input fn that trolls you upon every 8th call.
    >>> f = hof8(max)
    >>> f(1, 2) + f(f(f(f(f(f(3, 4), 5), 6), 7), 8), 9) # 7 calls
    11
    >>> f(10, 11) # 8th call
    gr8 m8 i r8 8/8
    11
    >>> f(12, 13) # 9th call; message should not be printed
    13
    """
    callCount = [0] # this is a list; update w/ callCount[0] = <whatever>
    
    ______________________________________________________________________

    ______________________________________________________________________

    ______________________________________________________________________

    ______________________________________________________________________
    
    ______________________________________________________________________
    
    return _______________________________________________________________
\end{lstlisting}

---

\textbf{Of note}: there is a reason I've made \texttt{callCount} a list and not just a number! If the line were instead \texttt{callCount = 0}, we would not be able to reassign \texttt{callCount} within an inner function -- \textit{because assignments are made within the current frame}, even if the relevant name is already defined inside a parent frame. This is important to understand.

\end{enumerate}
\end{document}
