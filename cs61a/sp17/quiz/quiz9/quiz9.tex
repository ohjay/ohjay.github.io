\documentclass[twoside]{article}
\usepackage{../quiz}

\pagestyle{myheadings}

\lstset{
    language=Python,
    basicstyle=\ttfamily,
    showstringspaces=false
    keywordstyle=\color{black},
    commentstyle=\color{black},
    stringstyle=\color{black},
}

\def\semester{Spring 2017}

% \newcommand{\solution}[1]{{\color{red}#1}}
\newcommand\solution[1]{}

%%% Actual content begins here %%%
\title{\sc Discussion Quiz 9 \solution{Solutions}}

\begin{document}
\thispagestyle{empty}
\maketitle

\begin{enumerate}
%%% Question 1: Pin the Tail %%%
\q{3}{Heads or Tails}

Identify whether or not each of the following procedures uses a constant amount of space in a tail-recursive Scheme implementation (i.e. whether \textbf{every} recursive call is a tail call).

%%% Q1a %%%
\begin{lstlisting}
(define (copy lst result)
    (if (null? lst) result
        ((lambda (copy) copy) (copy (cdr lst)
                                    (append result (list (car lst)))))))
\end{lstlisting}

(Remember that {\tt append} takes zero or more lists and constructs a new list with all of the lists' elements.)
\begin{lstlisting}

__________________________________________________________________________
\end{lstlisting}

%%% Q1b %%%
\begin{lstlisting}
(define (broken lst) (broken (broken lst)))
\end{lstlisting}
\vspace{0.21cm}
\begin{lstlisting}
__________________________________________________________________________
\end{lstlisting}

%%% Q1c %%%
\begin{lstlisting}
(define (is-ascending lst last-num)
    (if (null? lst) #t
        (and (is-ascending (cdr lst) (car lst)) (> (car lst) last-num))))
\end{lstlisting}

(Assume that this procedure is always called with a {\tt last-num} that is less than all of the elements in the list.)
\begin{lstlisting}

__________________________________________________________________________
\end{lstlisting}

%%% Question 2: Hail Recursion %%%
\q{4}{Hail Recursion}

Write a \emph{tail-recursive} version of {\tt hailstone}. This procedure accepts a positive integer {\tt n} and returns a list that contains the hailstone sequence starting at {\tt n}. For instance, {\tt (hailstone 5)} would return {\tt (5 16 8 4 2 1)}.

\begin{lstlisting}
(define (hailstone n)
  (define (hs-helper n lst)
  
  ________________________________________________________________________
  
  ________________________________________________________________________
    
  ________________________________________________________________________)
  
  ________________________________________________________________________)
\end{lstlisting}

%%% Question 3: Humans Need Not Apply %%%
\q{3}{Humans Need Not Apply}

What does {\tt eval} do, in the context of an interpreter? What does {\tt apply} do?

\end{enumerate}
\end{document}
