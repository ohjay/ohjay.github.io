\documentclass[twoside]{article}
\usepackage{../quiz}
\usepackage{fancyhdr}

\pagestyle{fancy}
\renewcommand{\headrulewidth}{0pt}
\cfoot{}
\rfoot{Solutions at \textbf{\href{http://owenjow.xyz/cs61a/section-quizzes/}{owenjow.xyz/cs61a/section-quizzes}}. Credit to Brian Hou for question material.}
\renewcommand{\footrulewidth}{0.4pt}

\lstset{
    language=Python,
    basicstyle=\ttfamily,
    showstringspaces=false
    keywordstyle=\color{black},
    commentstyle=\color{black},
    stringstyle=\color{black},
    escapeinside={<*}{*>},
    moredelim=**[is][\color{red}]{@}{@},
}

\def\semester{Spring 2017}
\newcommand{\solution}[1]{{\color{red}#1}}

%%% Actual, flexible content begins here %%%
\title{\sc Discussion Quiz 10 \solution{Solutions}}

\begin{document}
\maketitle

\begin{enumerate}
%%% Q1 %%%
\q{5}{Extreme Stream Supreme (translated from Scheme)}

What are the first four values of the stream \lstinline{s}?
\vspace{0.1in}

\begin{lstlisting}
def sweet(dreams):
    return Stream(Link(dreams), lambda: sweet(Link(dreams)))

def mix(tape):
    pre = match(lambda x: x.rest, tape.rest)
    temp = pre
    while temp.rest is not Link.empty:
        temp = temp.rest
    temp.rest = Link(tape.first.rest)
    return Stream(tape.first.first, lambda: mix(pre))

def match(dot, com):
    if com is Link.empty:
        return com
    return Link(dot(com.first), match(dot, com.rest))

s = mix(match(sweet, Link(1, Link(2, Link(3)))))
\end{lstlisting}

\:
\solution{\lstinline{Link(1), Link(Link(2)), Link(Link(Link(3))), Link(Link(Link(Link(1))))}}

\newpage

%%% Q2 %%%
\q{5}{Tree Traversal}

Make \lstinline{BinTree}s iterable! such that if the tree were a binary \textit{search} tree, we would iterate over the values in order of least to greatest. In other words, we want to perform an \textit{inorder traversal}, in which we iterate over the labels on the \lstinline{left}... followed by the current node's label... followed by the labels on the \lstinline{right}.

What method(s) do we need to implement?
\vspace{0.1in}

\begin{lstlisting}
class BinTree:
    empty = ()
    def __init__(self, label, left=empty, right=empty):
        self.label = label
        self.left = left
        self.right = right

    @def __iter__(self):
        yield from self.left
        yield self.label
        yield from self.right
        
    Or, equivalently:
    
    def __iter__(self):
        for l_label in self.left:
            yield l_label
        yield self.label
        for r_label in self.right:
            yield r_label@
\end{lstlisting}

\end{enumerate}
\end{document}
