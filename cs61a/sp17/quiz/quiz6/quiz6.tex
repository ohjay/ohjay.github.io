\documentclass[twoside]{article}
\usepackage{../quiz}
\usepackage{fancyhdr}

\pagestyle{fancy}
\renewcommand{\headrulewidth}{0pt}
\cfoot{}
\rfoot{Solutions at \textbf{\href{http://owenjow.xyz/cs61a/section-quizzes/}{owenjow.xyz/cs61a/section-quizzes}}. Credit to Kristin Stephens-Martinez for the questions.}
\renewcommand{\footrulewidth}{0.4pt}

\lstset{
    language=Python,
    basicstyle=\ttfamily,
    showstringspaces=false
    keywordstyle=\color{black},
    commentstyle=\color{black},
    stringstyle=\color{black},
    escapeinside={<*}{*>},
}

\def\semester{Spring 2017}

% \newcommand{\solution}[1]{{\color{red}#1}}
\newcommand\solution[1]{}

%%% Actual, flexible content begins here %%%
\title{\sc Quiz 6 \solution{Solutions}}

\begin{document}
\maketitle

\begin{enumerate}
%%% Q1 %%%
\q{10}{I've had enough of foobar}

For each row below, write the output displayed by the interactive Python
interpreter when the expression is evaluated. Expressions are evaluated in
order, and \textbf{expressions may affect later expressions}.

If nothing is output, write {\sc Nothing}.
If the interpreter would report an error, write {\sc Error}.
You \emph{should} include any lines displayed before an error.
Assume that you have started Python 3 and executed the following:
\vspace{0.1in}

\begin{lstlisting}
class Foo:
    def __init__(self, n):
        self.n = n
        self.curr = 0
    def __next__(self):
        self.curr = (self.curr + 1) % self.n
        return self.curr
    def __iter__(self):
        return self
    
def fighters(m):
    print('a')
    it, x = Foo(4), 0
    while x < m:
        print('b')
        yield x
        print('c')
        x += next(iter(it))
\end{lstlisting}

\begin{table}[h!]\large
\newcolumntype{M}[1]{>{\centering\arraybackslash}m{#1}}
\begin{tabular}{cc}

%%% Left table %%%
\begin{tabular}[t]{|l|M{1.3675in}|}
\hline
\multicolumn{1}{|l|}{\bf Expression} & {\bf Interactive Output} \\
\hline
\lstinline$print('d')$ & \solution{\begin{tabular}{M{2.5cm}} \rule{0pt}{4.9ex} \\ \tt d\end{tabular}} \\ [8ex]
\hline
\lstinline$b = fighters(7)$ & \solution{\begin{tabular}{M{2.5cm}} \rule{0pt}{4.9ex} \\ \tt Nothing\end{tabular}} \\ [8ex]
\hline
\lstinline$print('e', next(b))$ & \solution{\begin{tabular}{M{2.5cm}} \rule{0pt}{2.85ex} \\ \tt a \\ \tt b \\ \tt e 0\end{tabular}} \\ [8ex]
\hline
\end{tabular}

%%% Right table %%%
\begin{tabular}[t]{|l|M{1.3675in}|}
\hline
\multicolumn{1}{|l|}{\bf Expression} & {\bf Interactive Output} \\
\hline
\lstinline$print('f', next(b))$ & \solution{\begin{tabular}{M{2.5cm}} \rule{0pt}{2.6ex} \\ \tt c \\ \tt b \\ \tt f 1\end{tabular}} \\ [8ex]
\hline
\lstinline$print([x for x in b])$ & \solution{\begin{tabular}{M{2.5cm}} \\ \tt c \\ \tt b \\ \tt c \\ \tt b \\ \tt c \\ \tt b \\ \tt c \\ \tt [3, 6, 6] \\ \\\end{tabular}} \\ [18.82ex]
\hline
\end{tabular}

\end{tabular}
\end{table}

\end{enumerate}
\end{document}
