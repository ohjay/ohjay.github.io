\documentclass[twoside]{article}
\usepackage{../quiz}
\usepackage{fancyhdr}

\pagestyle{fancy}
\renewcommand{\headrulewidth}{0pt}
\cfoot{}
\rfoot{Solutions at \textbf{\href{http://owenjow.xyz/cs61a/section-quizzes/}{owenjow.xyz/cs61a/section-quizzes}}. Credit to Brian Hou for the second question.}
\renewcommand{\footrulewidth}{0.4pt}

\lstset{
    language=Python,
    basicstyle=\ttfamily,
    showstringspaces=false
    keywordstyle=\color{black},
    commentstyle=\color{black},
    stringstyle=\color{black},
    escapeinside={<*}{*>},
}

\def\semester{Spring 2017}

% \newcommand{\solution}[1]{{\color{red}#1}}
\newcommand\solution[1]{}

%%% Actual, flexible content begins here %%%
\title{\sc Quiz 7 \solution{Solutions}}

\begin{document}
\maketitle

\begin{enumerate}
%%% Q1 %%%
\q{4}{As a function of \lstinline{n}}

Identify the order of growth of the runtime as a function of \lstinline{n}.
\vspace{0.1in}

\begin{lstlisting}
def f0(n):
    while n > 0:
        print('print(print(None)))')
        n -= 1
\end{lstlisting}
~\\
\lstinline{_______________}

\begin{lstlisting}
def f1(n):
    while n > 0:
        print('make')
        n -= 2
\end{lstlisting}
~\\
\lstinline{_______________}

\begin{lstlisting}
def f2(n):
    while n > 0:
        print('lemonade')
        n //= 2
\end{lstlisting}
~\\
\lstinline{_______________}

\newpage

%%% Q2 %%%
\q{6}{\textit{All} of the paths}

Write a function \lstinline{all_paths} that takes in a \lstinline{Tree} and returns a list of paths from the root to leaves. Each path should be represented as a list.

\begin{lstlisting}
def all_paths(tree):
\end{lstlisting}

\end{enumerate}
\end{document}
