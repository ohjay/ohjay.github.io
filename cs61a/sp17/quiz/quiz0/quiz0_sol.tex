\documentclass[twoside]{article}
\usepackage{../quiz}
\usepackage{fancyhdr}

\pagestyle{fancy}
\renewcommand{\headrulewidth}{0pt}
\cfoot{}
\rfoot{Solutions to these quizzes will be posted at \textbf{\href{http://owenjow.xyz/cs61a/section-quizzes/}{owenjow.xyz/cs61a/section-quizzes}}.}
\renewcommand{\footrulewidth}{0.4pt}

\lstset{
    language=Python,
    basicstyle=\ttfamily,
    showstringspaces=false
    keywordstyle=\color{black},
    commentstyle=\color{black},
    stringstyle=\color{black},
    escapeinside={<*}{*>},
}

\def\semester{Spring 2k17}
\newcommand{\solution}[1]{{\color{red}#1}}

%%% Actual, flexible content begins here %%%
\title{\sc Discussion Quiz 0 \solution{Solutions}}

\begin{document}
\maketitle

\begin{enumerate}
%%% Q1 %%%
\q{6}{The Science of Computers}
\begin{enumerate}
\item What is a computer scientist? What is a programmer? What skills do they need?

\solution{Potential answer: a computer scientist is someone who studies everything about computers -- in particular the things they make possible. A programmer is someone who \textit{programs} the computers, i.e. someone who actually makes them do all of those ``possible" things. 

To succeed as a computer scientist, the most important skills are perhaps logic, resourcefulness, and tenacity. \textit{All of these} can be learned and developed, no matter where you are right now. However, you will often need tenacity in order to successfully develop everything else -- so in this class we tend to push the idea of tenacity the most.}
\newline

\item How should you learn in this class? What is necessary for success?

\solution{Potential answer: you learn by reading/hearing/absorbing information, and then \textit{doing}. For success, you'll want to keep pace with the course content (and do whatever it takes to really understand it), start assignments early, and ask questions as they arise via Piazza/OH/your classmates. You'll also want to build up as much love/enthusiasm for the material as possible.}
\newline

\item What are some of the advantages and disadvantages to working with others?

\solution{Some advantages: access to more skills/ideas/perspectives, more fun, more support, increased liability, the fact that putting your thoughts into words helps you understand things more clearly, ...

Some disadvantages: differing perspectives, division of labor being tough to manage, other people are annoying, ...

Overall, we'd like to encourage teamwork. It's generally more beneficial than detrimental, and it's for the most part how you'll be working in the real world. Plus, you can usually do better and cooler things when you have more people on your team.}
\newline

\end{enumerate}

%%% Q2 %%%
\q{2}{There is an island with three women. One woman has a red dot on her forehead and the other two women have blue dots on their foreheads.} \newline\newline If a woman ever learns the color of the dot on her forehead, she must permanently leave the island in the middle of that night. \newline\newline One day, an oracle appears and says ``at least one woman has a blue dot on her forehead." The women are all aware that the oracle speaks the truth. Furthermore, all of the women are perfect logicians (and know that the others are perfect logicians too). How many nights does it take for everyone to leave the island?

\solution{Three nights.}
\newline

%%% Q3 %%%
\q{2}{There is an island with 100 women. 50 of the women have red dots on their foreheads, and the other 50 women have blue dots on their foreheads... (the rest is the same as Q2).}

\solution{51 nights.}

\end{enumerate}
\end{document}
