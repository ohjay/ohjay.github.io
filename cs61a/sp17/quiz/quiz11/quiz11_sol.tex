\documentclass[twoside]{article}
\usepackage{../quiz}
\usepackage{fancyhdr}

\pagestyle{fancy}
\renewcommand{\headrulewidth}{0pt}
\cfoot{}
\rfoot{Solutions at \textbf{\href{http://owenjow.xyz/cs61a/section-quizzes/}{owenjow.xyz/cs61a/section-quizzes}}. Credit to Brian Hou for the questions.}
\renewcommand{\footrulewidth}{0.4pt}

\lstset{
    language=Python,
    basicstyle=\ttfamily,
    showstringspaces=false
    keywordstyle=\color{black},
    commentstyle=\color{black},
    stringstyle=\color{black},
    escapeinside={<*}{*>},
    moredelim=**[is][\color{red}]{@}{@},
}

\def\semester{Spring 2017}
\newcommand{\solution}[1]{{\color{red}#1}}

%%% Actual, flexible content begins here %%%
\title{\sc Discussion Quiz 11 \solution{Solutions}}

\begin{document}
\maketitle

\begin{enumerate}
%%% Q1 %%%
\q{5}{The Powers that Three}

Write a recursive \lstinline{select} statement to compute the powers of three from $3^0, 3^1, ..., 3^8$. You should not need all of the lines provided (an ideal solution will only use two or three).
\vspace{0.1in}

\begin{lstlisting}
with threes(pwr) as (
  @select 1 union
  select pwr * 3 from threes where pwr <= 2187@
)
select pwr from threes;
\end{lstlisting}

\vspace{2in}

%%% Q2 %%%
\q{5}{Cats!... the second most popular pets in the US}

We can also use SQL to determine the anagrams of a word!
Let's try using SQL to find the anagrams of ``cats".
(Remember that string concatenation in SQL is performed using the \lstinline{||} operator.)
\vspace{0.1in}

\begin{lstlisting}
with given(char, weight) as (
  select 'c',    1 union
  select 'a',   10 union
  select 't',  100 union
  select 's', 1000
) 
select @a.char || b.char || c.char || d.char@
from @given as a, given as b, given as c, given as d@
where @a.weight + b.weight + c.weight + d.weight = 1111@;
\end{lstlisting}

\end{enumerate}
\end{document}
